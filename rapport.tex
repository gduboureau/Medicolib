\documentclass[a4paper]{book}
\usepackage{fullpage}

\usepackage[utf8]{inputenc}
\usepackage[T1]{fontenc}
\usepackage[french]{babel}
\usepackage{graphicx}
\usepackage{titlesec}

\begin{document}

\title{Conception et Réalisation d’une application web pour la gestion
d’un cabinet médical}
\author{Loustau Valentin \and Duboureau Guillaume \and Ephrem Jennifer \and Abdoul Goudoussy Diallo}
\date{}
\maketitle
\let\cleardoublepage\clearpage
\tableofcontents

\chapter{Introduction}

\chapter{Analyse des besoins}
\section{Identification des acteurs}

Un acteur représente un rôle joué par une personne externe ou par un processus qui
interagit avec le système [13].
\newline\newline
Les acteurs de notre système sont :
\newline
\begin{itemize}
    \item[$\bullet$] \textbf{Patient} : il s’agit d’un acteur qui utilise le site pour gérer 
    (ajouter, consulter, supprimer) ses rendez-vous, consulter son dossier médical. 
    Il peut également mettre à jour ses informations personnelles et communiquer ses infos au médecin.
    \item[$\bullet$] \textbf{Médecin} : il s’agit d’un acteur qui gère les dossiers des patients, prescrit des ordonnances et 
    les imprime. Il s’occupe de la gestion des rendez-vous de ses patients et les notifie s’il effectue 
    quelconque modification.
  \end{itemize}
  
\section{Les besoins fonctionnels}
\subsection{Pour le patient}
\begin{enumerate}
    \item Création de compte\newline
        \begin{itemize}
            \item[$\bullet$] \textbf{\underline{Quantifications}}: Pouvoir créer qu’un seul compte par adresse mail.
            \item[$\bullet$] \textbf{\underline{Contraintes ou difficultés techniques}}: Sécurité des données personnelles des patients.
            \item[$\bullet$] \textbf{\underline{Énonciation des risques et parades}}: Risque de vols des données.\newline
            \textbf{Parade}: Protocole de sécurité pour protéger les données.
            \item[$\bullet$] \textbf{\underline{Spécification des tests de contrôle}}: Tests unitaires.\newline
        \end{itemize}

    \item Gestion et consultation des rendez-vous en ligne\newline
    \begin{itemize}
        \item[$\bullet$] \textbf{\underline{Quantifications}}: Pouvoir prendre plusieurs rendez-vous selon les créneaux libres sur le planning. 
        De plus, possibilité de visualiser ses rendez-vous en temps réel.
        \item[$\bullet$] \textbf{\underline{Éléments de faisabilité}}: Mettre en place un système de gestion de rendez-vous en ligne tel que doctolib et vérifier sa faisabilité.
        \item[$\bullet$] \textbf{\underline{Contraintes ou difficultés techniques}}: Assurer la synchronisation en temps réel avec le calendrier du médecin, garantir la disponibilité d’un rendez-vous.
        \item[$\bullet$] \textbf{\underline{Énonciation des risques et parades}}: Chevauchement de prise de rendez-vous pour des patients différents, ou sélectionner un rendez-vous qui n'existe plus.\newline
        \textbf{Parade}: Bloquer la prise d’un rendez-vous pour le premier arrivé.
        \item[$\bullet$] \textbf{\underline{Spécification des tests de contrôle}}: Tests unitaires.
    \end{itemize}
        
\end{enumerate}

\subsection{Pour le médecin}

\begin{enumerate}
    \item Gérer (ajouter/modifier/supprimer documents) les dossiers médicaux des patients \newline
    \begin{itemize}
        \item[$\bullet$] \textbf{\underline{Quantifications}}: Vue complète des dossiers des patients en temps réel.

        \item[$\bullet$] \textbf{\underline{Éléments de faisabilité}}: Évaluation de différents systèmes de dossiers médicaux pour comprendre les fonctionnalités
        \item[$\bullet$] \textbf{\underline{Contraintes ou difficultés techniques}}: Sécurité des données 
        \item[$\bullet$] \textbf{\underline{Énonciation des risques et parades}}: Risques de pertes de données à cause d’un souci technique, risque de non synchronisation/mise à jour
        \textbf{Parade}: Historique de sauvegarde
        \item[$\bullet$] \textbf{\underline{Spécification des tests de contrôle}}: Tests de sécurité.\newline
    \end{itemize}

    \item Gestion de ses rendez-vous en ligne\newline
    \begin{itemize}
        \item[$\bullet$] \textbf{\underline{Quantification}}: Possibilité de visualiser son calendrier de rendez-vous en temps réel.
        \item[$\bullet$] \textbf{\underline{Éléments de faisabilité}}: Mettre en place un système de gestion de rendez-vous en ligne tel que doctolib et vérifier sa faisabilité.
        \item[$\bullet$] \textbf{\underline{Contraintes ou difficultés techniques}}: Assurer la synchronisation en temps réel avec le calendrier, garantir la disponibilité d’un rendez-vous 
        (ne pas placer le rendez-vous sur le créneau d’un autre patient, dû à une synchronisation non effectué).
        \item[$\bullet$] \textbf{\underline{Énonciation des risques et parades}}: Rendez-vous en simultané pour des patients différents, sélectionner un rendez-vous qui n'existe plus.\newline
        \textbf{Parade}:  Bloquer la prise d’un rendez-vous pour le premier arrivé.
        \item[$\bullet$] \textbf{\underline{Spécification des tests de contrôle}}: Tests unitaires. \newline
    \end{itemize}

    \item Communication avec les patients\newline
    \begin{itemize}
        \item[$\bullet$] \textbf{\underline{Éléments de faisabilité}}: Mise en place d’un système de notifications par e-mail. 
        \item[$\bullet$] \textbf{\underline{Contraintes ou difficultés techniques}}: Aucune contrainte. \newline
    \end{itemize}

    \item Consulter son planning des rendez-vous\newline
    \begin{itemize}
        \item[$\bullet$] \textbf{\underline{Quantification}}: Le médecin doit pouvoir consulter son planning pour chaque jour de la semaine, ainsi que pour une période donnée : 1 semaine.
		Le planning doit être mis à jour en temps réel lorsque des rendez-vous sont ajoutés ou supprimés.
        \item[$\bullet$] \textbf{\underline{Éléments de faisabilité}}: La consultation du planning peut être réalisée en utilisant une interface graphique simple, qui affiche les rendez-vous (stockés dans une base de données) 
        en fonction de l'heure et de la date tel que le logiciel Google Calendar.
        \item[$\bullet$] \textbf{\underline{Contraintes ou difficultés techniques}}: Garantir la confidentialité des informations stockées dans la base de données.
        \item[$\bullet$] \textbf{\underline{Énonciation des risques et parades}}:  Risque que les rendez-vous soient compromis si la base de données est piratée. Risque que le planning du médecin ne soit pas à jour après modification.\newline
        \textbf{Parade}: Le système peut être conçu pour utiliser des méthodes de cryptage pour protéger les données sensibles. Actualisation du planning après chaque modification.
        \item[$\bullet$] \textbf{\underline{Spécification des tests de contrôle}}: Des tests de validation peuvent être effectués pour vérifier que le système affiche correctement les rendez-vous dans l'interface utilisateur. Des tests de contrôle peuvent être effectués pour vérifier que les données sont correctement stockées dans la base de données et affichées dans l'interface utilisateur.\newline
    \end{itemize} 

    \item Bilan de santé (Le système doit assurer l’impression des fiches malades et les bilans): \newline
    \begin{itemize}
        \item[$\bullet$] \textbf{\underline{Contraintes ou difficultés techniques}}: Les fichiers doivent être au format PDF exclusivement pour ne pas perdre d’informations et pour une meilleure impression.
        \item[$\bullet$] \textbf{\underline{Énonciation des risques et parades}}: Risque de mauvais transfert/perte de données lors de l’envoie du bilan par le médecin au patient\newline
        \textbf{Parade}: Nous pourrons vérifier si le document reçu par le patient est vide (null) ou non.
    \end{itemize}
\end{enumerate}

\section{Les besoins fonctionnels}

Ce sont des besoins en relation avec la performance du système, la facilité d’utilisation,
l’ergonomie des interfaces, la sécurité etc. Et parmi ces besoins nous citons :

\begin{enumerate}
    \item Sécurité des données médicales des patients\newline
    \begin{itemize}
        \item[$\bullet$] \textbf{\underline{Éléments de faisabilité}}: Mise en place d’un système d’identification et d’authentification de chaque utilisateur (patients/médecins) pour garantir que seuls les utilisateurs autorisés aient accès au système et aux données sensibles.
        \item[$\bullet$] \textbf{\underline{Contraintes ou difficultés techniques}}: Complexité de la gestion des autorisations d'accès pour les différents utilisateurs.
        \item[$\bullet$] \textbf{\underline{Énonciation des risques et parades}}: Risque de faille et accès à un intrus aux données des patients.\newline
        \textbf{Parade}: Il faudrait pour cela protéger ces données.
        \item[$\bullet$] \textbf{\underline{Spécification des tests de contrôle}}: Connexion au logiciel avec les différents acteurs et vérifier leur autorisation (un patient n’a pas accès au rendez-vous de tous les patients par exemple). \newline
    \end{itemize}

\item Simplicité et ergonomie de l’interface graphique\newline
\begin{itemize}
    \item[$\bullet$] \textbf{\underline{Éléments de faisabilité}}: Mise en place d’une interface intuitive et facile à utiliser pour les utilisateurs, 
    en utilisant des icônes, des boutons et d'autres éléments de conception familiers (système de navigation logique), avec des choix judicieux de couleurs, 
    de polices et d’images pour renforcer la clarté et la lisibilité de l'interface.
    \item[$\bullet$] \textbf{\underline{Contraintes ou difficultés techniques}}: Temps nécessaire pour trouver le juste équilibre entre simplicité et fonctionnalité 
    mais aussi pour la conception et la mise en œuvre d’une telle interface.
    \item[$\bullet$] \textbf{\underline{Énonciation des risques et parades}}: Risque de confusion pour les utilisateurs et bug de navigation.\newline
    \textbf{Parade}: Il faudrait faire tester le logiciel par plusieurs utilisateurs et recueillir un feedback.
    \item[$\bullet$] \textbf{\underline{Spécification des tests de contrôle}}: Tests unitaires.	\newline
\end{itemize}

\item Performance du système en temps de réponse, stockage mémoire

\begin{itemize}
    \item[$\bullet$] \textbf{\underline{Quantification}}: Temps de réponses en fonction des différentes actions (pour une requête utilisateur, pour le chargement d'une page).
    \item[$\bullet$] \textbf{\underline{Éléments de faisabilité}}: Un essai peut être implémenté pour mesurer la performance actuelle du système et trouver des améliorations à faire.
    \item[$\bullet$] \textbf{\underline{Contraintes ou difficultés techniques}}: Posséder un stockage suffisant pour contenir toutes les données enregistrées, réussir à élaborer des algorithmes pour exécuter les requêtes en temps réel.
    \item[$\bullet$] \textbf{\underline{Énonciation des risques et parades}}:  Les risques peuvent être un stockage insuffisant en mémoire.\newline
    \textbf{Parade}: Utilisation de technologies de stockage plus performantes.
    \item[$\bullet$] \textbf{\underline{Spécification des tests de contrôle}}: Tests de performance pour mesurer les temps de réponse, tests de stockage pour vérifier que la mémoire est suffisante.
\end{itemize} 

\end{enumerate}

\chapter{Conception}
\chapter{Réalisation}
\chapter{Tests}
\chapter{Résultats}
\chapter{Conclusion}

\end{document}